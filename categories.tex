\IfFileExists{stacks-project.cls}{%
\documentclass{stacks-project}
}{%
\documentclass{amsart}
}

% For dealing with references we use the comment environment
\usepackage{verbatim}
\newenvironment{reference}{\comment}{\endcomment}
%\newenvironment{reference}{}{}
\newenvironment{slogan}{\comment}{\endcomment}
\newenvironment{history}{\comment}{\endcomment}

% For commutative diagrams we use Xy-pic
\usepackage[all]{xy}

% We use 2cell for 2-commutative diagrams.
\xyoption{2cell}
\UseAllTwocells

% We use multicol for the list of chapters between chapters
\usepackage{multicol}

% This is generally recommended for better output
\usepackage{lmodern}
\usepackage[T1]{fontenc}

% For cross-file-references
\usepackage{xr-hyper}

% Package for hypertext links:
\usepackage{hyperref}

% For any local file, say "hello.tex" you want to link to please
% use \externaldocument[hello-]{hello}
\externaldocument[introduction-]{introduction}
\externaldocument[conventions-]{conventions}
\externaldocument[sets-]{sets}
\externaldocument[categories-]{categories}
\externaldocument[topology-]{topology}
\externaldocument[sheaves-]{sheaves}
\externaldocument[sites-]{sites}
\externaldocument[stacks-]{stacks}
\externaldocument[fields-]{fields}
\externaldocument[algebra-]{algebra}
\externaldocument[brauer-]{brauer}
\externaldocument[homology-]{homology}
\externaldocument[derived-]{derived}
\externaldocument[simplicial-]{simplicial}
\externaldocument[more-algebra-]{more-algebra}
\externaldocument[smoothing-]{smoothing}
\externaldocument[modules-]{modules}
\externaldocument[sites-modules-]{sites-modules}
\externaldocument[injectives-]{injectives}
\externaldocument[cohomology-]{cohomology}
\externaldocument[sites-cohomology-]{sites-cohomology}
\externaldocument[dga-]{dga}
\externaldocument[dpa-]{dpa}
\externaldocument[sdga-]{sdga}
\externaldocument[hypercovering-]{hypercovering}
\externaldocument[schemes-]{schemes}
\externaldocument[constructions-]{constructions}
\externaldocument[properties-]{properties}
\externaldocument[morphisms-]{morphisms}
\externaldocument[coherent-]{coherent}
\externaldocument[divisors-]{divisors}
\externaldocument[limits-]{limits}
\externaldocument[varieties-]{varieties}
\externaldocument[topologies-]{topologies}
\externaldocument[descent-]{descent}
\externaldocument[perfect-]{perfect}
\externaldocument[more-morphisms-]{more-morphisms}
\externaldocument[flat-]{flat}
\externaldocument[groupoids-]{groupoids}
\externaldocument[more-groupoids-]{more-groupoids}
\externaldocument[etale-]{etale}
\externaldocument[chow-]{chow}
\externaldocument[intersection-]{intersection}
\externaldocument[pic-]{pic}
\externaldocument[weil-]{weil}
\externaldocument[adequate-]{adequate}
\externaldocument[dualizing-]{dualizing}
\externaldocument[duality-]{duality}
\externaldocument[discriminant-]{discriminant}
\externaldocument[derham-]{derham}
\externaldocument[local-cohomology-]{local-cohomology}
\externaldocument[algebraization-]{algebraization}
\externaldocument[curves-]{curves}
\externaldocument[resolve-]{resolve}
\externaldocument[models-]{models}
\externaldocument[functors-]{functors}
\externaldocument[equiv-]{equiv}
\externaldocument[pione-]{pione}
\externaldocument[etale-cohomology-]{etale-cohomology}
\externaldocument[proetale-]{proetale}
\externaldocument[relative-cycles-]{relative-cycles}
\externaldocument[more-etale-]{more-etale}
\externaldocument[trace-]{trace}
\externaldocument[crystalline-]{crystalline}
\externaldocument[spaces-]{spaces}
\externaldocument[spaces-properties-]{spaces-properties}
\externaldocument[spaces-morphisms-]{spaces-morphisms}
\externaldocument[decent-spaces-]{decent-spaces}
\externaldocument[spaces-cohomology-]{spaces-cohomology}
\externaldocument[spaces-limits-]{spaces-limits}
\externaldocument[spaces-divisors-]{spaces-divisors}
\externaldocument[spaces-over-fields-]{spaces-over-fields}
\externaldocument[spaces-topologies-]{spaces-topologies}
\externaldocument[spaces-descent-]{spaces-descent}
\externaldocument[spaces-perfect-]{spaces-perfect}
\externaldocument[spaces-more-morphisms-]{spaces-more-morphisms}
\externaldocument[spaces-flat-]{spaces-flat}
\externaldocument[spaces-groupoids-]{spaces-groupoids}
\externaldocument[spaces-more-groupoids-]{spaces-more-groupoids}
\externaldocument[bootstrap-]{bootstrap}
\externaldocument[spaces-pushouts-]{spaces-pushouts}
\externaldocument[spaces-chow-]{spaces-chow}
\externaldocument[groupoids-quotients-]{groupoids-quotients}
\externaldocument[spaces-more-cohomology-]{spaces-more-cohomology}
\externaldocument[spaces-simplicial-]{spaces-simplicial}
\externaldocument[spaces-duality-]{spaces-duality}
\externaldocument[formal-spaces-]{formal-spaces}
\externaldocument[restricted-]{restricted}
\externaldocument[spaces-resolve-]{spaces-resolve}
\externaldocument[formal-defos-]{formal-defos}
\externaldocument[defos-]{defos}
\externaldocument[cotangent-]{cotangent}
\externaldocument[examples-defos-]{examples-defos}
\externaldocument[algebraic-]{algebraic}
\externaldocument[examples-stacks-]{examples-stacks}
\externaldocument[stacks-sheaves-]{stacks-sheaves}
\externaldocument[criteria-]{criteria}
\externaldocument[artin-]{artin}
\externaldocument[quot-]{quot}
\externaldocument[stacks-properties-]{stacks-properties}
\externaldocument[stacks-morphisms-]{stacks-morphisms}
\externaldocument[stacks-limits-]{stacks-limits}
\externaldocument[stacks-cohomology-]{stacks-cohomology}
\externaldocument[stacks-perfect-]{stacks-perfect}
\externaldocument[stacks-introduction-]{stacks-introduction}
\externaldocument[stacks-more-morphisms-]{stacks-more-morphisms}
\externaldocument[stacks-geometry-]{stacks-geometry}
\externaldocument[moduli-]{moduli}
\externaldocument[moduli-curves-]{moduli-curves}
\externaldocument[examples-]{examples}
\externaldocument[exercises-]{exercises}
\externaldocument[guide-]{guide}
\externaldocument[desirables-]{desirables}
\externaldocument[coding-]{coding}
\externaldocument[obsolete-]{obsolete}
\externaldocument[fdl-]{fdl}
\externaldocument[index-]{index}

% Theorem environments.
%
\theoremstyle{plain}
\newtheorem{theorem}[subsection]{Theorem}
\newtheorem{proposition}[subsection]{Proposition}
\newtheorem{lemma}[subsection]{Lemma}

\theoremstyle{definition}
\newtheorem{definition}[subsection]{Definition}
\newtheorem{example}[subsection]{Example}
\newtheorem{exercise}[subsection]{Exercise}
\newtheorem{situation}[subsection]{Situation}

\theoremstyle{remark}
\newtheorem{remark}[subsection]{Remark}
\newtheorem{remarks}[subsection]{Remarks}

\numberwithin{equation}{subsection}

% Macros
%
\def\lim{\mathop{\mathrm{lim}}\nolimits}
\def\colim{\mathop{\mathrm{colim}}\nolimits}
\def\Spec{\mathop{\mathrm{Spec}}}
\def\Hom{\mathop{\mathrm{Hom}}\nolimits}
\def\Ext{\mathop{\mathrm{Ext}}\nolimits}
\def\SheafHom{\mathop{\mathcal{H}\!\mathit{om}}\nolimits}
\def\SheafExt{\mathop{\mathcal{E}\!\mathit{xt}}\nolimits}
\def\Sch{\mathit{Sch}}
\def\Mor{\mathop{\mathrm{Mor}}\nolimits}
\def\Ob{\mathop{\mathrm{Ob}}\nolimits}
\def\Sh{\mathop{\mathit{Sh}}\nolimits}
\def\NL{\mathop{N\!L}\nolimits}
\def\CH{\mathop{\mathrm{CH}}\nolimits}
\def\proetale{{pro\text{-}\acute{e}tale}}
\def\etale{{\acute{e}tale}}
\def\QCoh{\mathit{QCoh}}
\def\Ker{\mathop{\mathrm{Ker}}}
\def\Im{\mathop{\mathrm{Im}}}
\def\Coker{\mathop{\mathrm{Coker}}}
\def\Coim{\mathop{\mathrm{Coim}}}

% Boxtimes
%
\DeclareMathSymbol{\boxtimes}{\mathbin}{AMSa}{"02}

%
% Macros for moduli stacks/spaces
%
\def\QCohstack{\mathcal{QC}\!\mathit{oh}}
\def\Cohstack{\mathcal{C}\!\mathit{oh}}
\def\Spacesstack{\mathcal{S}\!\mathit{paces}}
\def\Quotfunctor{\mathrm{Quot}}
\def\Hilbfunctor{\mathrm{Hilb}}
\def\Curvesstack{\mathcal{C}\!\mathit{urves}}
\def\Polarizedstack{\mathcal{P}\!\mathit{olarized}}
\def\Complexesstack{\mathcal{C}\!\mathit{omplexes}}
% \Pic is the operator that assigns to X its picard group, usage \Pic(X)
% \Picardstack_{X/B} denotes the Picard stack of X over B
% \Picardfunctor_{X/B} denotes the Picard functor of X over B
\def\Pic{\mathop{\mathrm{Pic}}\nolimits}
\def\Picardstack{\mathcal{P}\!\mathit{ic}}
\def\Picardfunctor{\mathrm{Pic}}
\def\Deformationcategory{\mathcal{D}\!\mathit{ef}}


\begin{document}

\title{Categories}
\maketitle

\phantomsection
\label{section-phantom}
\hfill
\href{http://github.com/danimalabares/stack}{github.com/danimalabares/stack}

\tableofcontents

\section{Definitions}
\label{section-definition-categories}

\noindent
We recall the definitions, partly to fix notation.

\begin{definition}
\label{definition-category}
A {\it category} $\mathcal{C}$ consists of the following data:
\begin{enumerate}
\item A set of objects $\Ob(\mathcal{C})$.
\item For each pair $x, y \in \Ob(\mathcal{C})$ a set of morphisms
$\Mor_\mathcal{C}(x, y)$.
\item For each triple $x, y, z\in \Ob(\mathcal{C})$ a composition
map $ \Mor_\mathcal{C}(y, z) \times \Mor_\mathcal{C}(x, y)
\to \Mor_\mathcal{C}(x, z) $, denoted $(\phi, \psi) \mapsto
\phi \circ \psi$.
\end{enumerate}
These data are to satisfy the following rules:
\begin{enumerate}
\item For every element $x\in \Ob(\mathcal{C})$ there exists a
morphism $\text{id}_x\in \Mor_\mathcal{C}(x, x)$ such that
$\text{id}_x \circ \phi = \phi$ and $\psi \circ \text{id}_x = \psi $ whenever
these compositions make sense.
\item Composition is associative, i.e., $(\phi \circ \psi) \circ \chi =
\phi \circ ( \psi \circ \chi)$ whenever these compositions make sense.
\end{enumerate}
\end{definition}

\begin{definition}
\label{definition-functor}
A {\it functor} $F : \mathcal{A} \to \mathcal{B}$
between two categories $\mathcal{A}, \mathcal{B}$ is given by the
following data:
\begin{enumerate}
\item A map $F : \Ob(\mathcal{A}) \to \Ob(\mathcal{B})$.
\item For every $x, y \in \Ob(\mathcal{A})$ a map
$F : \Mor_\mathcal{A}(x, y) \to \Mor_\mathcal{B}(F(x), F(y))$,
denoted $\phi \mapsto F(\phi)$.
\end{enumerate}
These data should be compatible with composition and identity morphisms
in the following manner: $F(\phi \circ \psi) =
F(\phi) \circ F(\psi)$ for a composable pair $(\phi, \psi)$ of
morphisms of $\mathcal{A}$ and $F(\text{id}_x) = \text{id}_{F(x)}$.
\end{definition}

\section{Monomorphisms}
\label{section-monomorphisms-and-epimorphisms}

\begin{definition}
\label{definition-mono-epi}
Let $\mathcal{C}$ be a category and let $f : X \to Y$ be
a morphism of $\mathcal{C}$.
\begin{enumerate}
\item We say that $f$ is a {\it monomorphism} if for every object
$W$ and every pair of morphisms $a, b : W \to X$ such that
$f \circ a = f \circ b$ we have $a = b$.
\item We say that $f$ is an {\it epimorphism} if for every object
$W$ and every pair of morphisms $a, b : Y \to W$ such that
$a \circ f = b \circ f$ we have $a = b$.
\end{enumerate}
\end{definition}

\begin{definition}
\label{definition-presheaves-injective-surjective}
Let $\mathcal{C}$ be a category, and let $\varphi : \mathcal{F}
\to \mathcal{G}$ be a map of presheaves of sets.
\begin{enumerate}
\item We say that $\varphi$ is {\it injective} if for every object
$U$ of $\mathcal{C}$ the map $\varphi_U : \mathcal{F}(U)
\to \mathcal{G}(U)$ is injective.
\item We say that $\varphi$ is {\it surjective} if for every object
$U$ of $\mathcal{C}$ the map $\varphi_U : \mathcal{F}(U)
\to \mathcal{G}(U)$ is surjective.
\end{enumerate}
\end{definition}

\begin{lemma}
\label{lemma-mono-epi}
The injective (resp.\ surjective) maps defined above
are exactly the monomorphisms (resp.\ epimorphisms) of
$\textit{PSh}(\mathcal{C})$. A map is an isomorphism
if and only if it is both injective and surjective.
\end{lemma}

\section{Presheaves}
\label{section-presheaves}

\begin{definition}
\label{definition-presheaves-sets}
A {\it presheaf of sets} on $\mathcal{C}$ is a contravariant
functor from $\mathcal{C}$ to $\textit{Sets}$. {\it Morphisms
of presheaves} are natural transformations of functors.
The category
of presheaves of sets is denoted $\textit{PSh}(\mathcal{C})$
or $\hat{\mathcal{C}}$.
\end{definition}

\section{Yoneda lemma}
\label{section-Yoneda-lemma}

The Yoneda lemma says that the sections over $a$
of a presheaf $X$ can be described completely
as natural transformations between the 
presheaf $\Hom(-,a)$ and $X$.

\begin{definition}
\label{definition-Yoneda-embedding}
\begin{reference}
\cite[Definition 1.1.3]{Cisinsky}
\end{reference}
Let $A$ be a category.
The {\it Yoneda embedding} is the functor
$$
h:A \to \hat{A}
$$
whose value at an object $a$ of $A$ is the presheaf
$$
h_a=\Hom_A(-,a).
$$
\end{definition}

In other words, the evaluation of the presheaf
$h_a$ at an object $c$ of $A$ is the set of
maps from $c$ to $a$.

\begin{theorem}[Yoneda lemma]
\label{theorem-Yoneda-lemma}
\begin{reference}
\cite[Theorem 1.1.4]{Cisinsky}
\end{reference}
For any presheaf $X$ over $A$, there is a natural bijection
(in $\text{Sets}$ I think!!)
\begin{align*}
\Hom_{\widehat{A}}(h_a,X) &\xrightarrow{\simeq} X_a  \\
(h_a \xrightarrow{u}X) &\longmapsto u_a(1_a)
\end{align*}

\end{theorem}

\section{Internal Hom}
\label{section-internal-hom}

\noindent
{\bf Upshot.} Internal Hom is when the
Hom set of two objects in some category
is in also an object of the category.
Down-to-earth, that for two sheaves $\mathcal{F},\mathcal{G}$,
$U\mapsto \Hom(\mathcal{F}(U),\mathcal{G}(U)$ 
is also a sheaf, called $\SheafHom$.

\medskip\noindent
I start with Stacks Project approach.

Let $(X, \mathcal{O}_X)$ be a ringed space.
Let $\mathcal{F}$, $\mathcal{G}$ be $\mathcal{O}_X$-modules.
Consider the rule
$$
U \longmapsto \Hom_{\mathcal{O}_X|_U}(\mathcal{F}|_U, \mathcal{G}|_U).
$$
It follows from the discussion in Sheaves, Section
\ref{sheaves-section-glueing-sheaves} that this is a sheaf of
abelian groups. In addition, given an element
$\varphi \in \Hom_{\mathcal{O}_X|_U}(\mathcal{F}|_U, \mathcal{G}|_U)$
and a section $f \in \mathcal{O}_X(U)$ then we can define
$f\varphi \in \Hom_{\mathcal{O}_X|_U}(\mathcal{F}|_U, \mathcal{G}|_U)$
by either precomposing with multiplication by $f$ on $\mathcal{F}|_U$
or postcomposing with multiplication by $f$ on $\mathcal{G}|_U$ (it gives
the same result). Hence we in fact get a sheaf of $\mathcal{O}_X$-modules.
We will denote this sheaf
$\SheafHom_{\mathcal{O}_X}(\mathcal{F}, \mathcal{G})$.
There is a canonical ``evaluation'' morphism
$$
\mathcal{F}
\otimes_{\mathcal{O}_X}
\SheafHom_{\mathcal{O}_X}(\mathcal{F}, \mathcal{G})
\longrightarrow
\mathcal{G}.
$$
For every $x \in X$ there is also a canonical morphism
$$
\SheafHom_{\mathcal{O}_X}(\mathcal{F}, \mathcal{G})_x
\to
\Hom_{\mathcal{O}_{X, x}}(\mathcal{F}_x, \mathcal{G}_x)
$$
which is rarely an isomorphism.

\medskip\noindent
{\bf Cartesian closed category} In the category of sets there is a bijection
$\Hom(X\times Y, Z)\cong \Hom(X, \text{Hom}(Y, Z))$ that depends
naturally on $X$, $Y$ and $Z$. The notions related to this bijection are
“Cartesian closed category”, “currying” and “internal Hom”.

\begin{definition}
\label{definition-Cartesian-closed-category}
A category $\mathcal{C}$ is {\it Cartesian closed} if:
\begin{enumerate}
\item $\mathcal{C}$ has all finite products 
(Caveat: some require that $\mathcal{C}$ has all finite limits)
\item For any object $Y$ the functor $- \times Y$ has a right adjoint,
which we will denote by $\text{Map}(Y,-)$ or by $-^Y$ .
\end{enumerate}
\end{definition}

\begin{remark}
\label{remark-currying}
By section 3 \href{https://ncatlab.org/nlab/show/internal+hom }{here}, the
second property above implies that we get a functor 
$\text{Map}(-,-) :\mathcal{C}^{\text{op}} \times \mathcal{C} \to \mathcal{C}$,
and moreover we get natural isomorphisms 
$\Hom(X, \text{Map}(Y, Z)) \cong \Hom(X \times Y, Z)$
and $\text{Map}(X, \text{Map}(Y, Z))\cong \text{Map}(X \times Y, Z)$.
\end{remark}



\bibliography{my}
\bibliographystyle{amsalpha}



\end{document}

