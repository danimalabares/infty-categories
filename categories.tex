\IfFileExists{stacks-project.cls}{%
\documentclass{stacks-project}
}{%
\documentclass{amsart}
}

% For dealing with references we use the comment environment
\usepackage{verbatim}
\newenvironment{reference}{\comment}{\endcomment}
%\newenvironment{reference}{}{}
\newenvironment{slogan}{\comment}{\endcomment}
\newenvironment{history}{\comment}{\endcomment}

% For commutative diagrams we use Xy-pic
\usepackage[all]{xy}

% We use 2cell for 2-commutative diagrams.
\xyoption{2cell}
\UseAllTwocells

% We use multicol for the list of chapters between chapters
\usepackage{multicol}

% This is generally recommended for better output
\usepackage{lmodern}
\usepackage[T1]{fontenc}

% For cross-file-references
\usepackage{xr-hyper}

% Package for hypertext links:
\usepackage{hyperref}

% For any local file, say "hello.tex" you want to link to please
% use \externaldocument[hello-]{hello}
\externaldocument[introduction-]{introduction}
\externaldocument[conventions-]{conventions}
\externaldocument[sets-]{sets}
\externaldocument[categories-]{categories}
\externaldocument[topology-]{topology}
\externaldocument[sheaves-]{sheaves}
\externaldocument[sites-]{sites}
\externaldocument[stacks-]{stacks}
\externaldocument[fields-]{fields}
\externaldocument[algebra-]{algebra}
\externaldocument[brauer-]{brauer}
\externaldocument[homology-]{homology}
\externaldocument[derived-]{derived}
\externaldocument[simplicial-]{simplicial}
\externaldocument[more-algebra-]{more-algebra}
\externaldocument[smoothing-]{smoothing}
\externaldocument[modules-]{modules}
\externaldocument[sites-modules-]{sites-modules}
\externaldocument[injectives-]{injectives}
\externaldocument[cohomology-]{cohomology}
\externaldocument[sites-cohomology-]{sites-cohomology}
\externaldocument[dga-]{dga}
\externaldocument[dpa-]{dpa}
\externaldocument[sdga-]{sdga}
\externaldocument[hypercovering-]{hypercovering}
\externaldocument[schemes-]{schemes}
\externaldocument[constructions-]{constructions}
\externaldocument[properties-]{properties}
\externaldocument[morphisms-]{morphisms}
\externaldocument[coherent-]{coherent}
\externaldocument[divisors-]{divisors}
\externaldocument[limits-]{limits}
\externaldocument[varieties-]{varieties}
\externaldocument[topologies-]{topologies}
\externaldocument[descent-]{descent}
\externaldocument[perfect-]{perfect}
\externaldocument[more-morphisms-]{more-morphisms}
\externaldocument[flat-]{flat}
\externaldocument[groupoids-]{groupoids}
\externaldocument[more-groupoids-]{more-groupoids}
\externaldocument[etale-]{etale}
\externaldocument[chow-]{chow}
\externaldocument[intersection-]{intersection}
\externaldocument[pic-]{pic}
\externaldocument[weil-]{weil}
\externaldocument[adequate-]{adequate}
\externaldocument[dualizing-]{dualizing}
\externaldocument[duality-]{duality}
\externaldocument[discriminant-]{discriminant}
\externaldocument[derham-]{derham}
\externaldocument[local-cohomology-]{local-cohomology}
\externaldocument[algebraization-]{algebraization}
\externaldocument[curves-]{curves}
\externaldocument[resolve-]{resolve}
\externaldocument[models-]{models}
\externaldocument[functors-]{functors}
\externaldocument[equiv-]{equiv}
\externaldocument[pione-]{pione}
\externaldocument[etale-cohomology-]{etale-cohomology}
\externaldocument[proetale-]{proetale}
\externaldocument[relative-cycles-]{relative-cycles}
\externaldocument[more-etale-]{more-etale}
\externaldocument[trace-]{trace}
\externaldocument[crystalline-]{crystalline}
\externaldocument[spaces-]{spaces}
\externaldocument[spaces-properties-]{spaces-properties}
\externaldocument[spaces-morphisms-]{spaces-morphisms}
\externaldocument[decent-spaces-]{decent-spaces}
\externaldocument[spaces-cohomology-]{spaces-cohomology}
\externaldocument[spaces-limits-]{spaces-limits}
\externaldocument[spaces-divisors-]{spaces-divisors}
\externaldocument[spaces-over-fields-]{spaces-over-fields}
\externaldocument[spaces-topologies-]{spaces-topologies}
\externaldocument[spaces-descent-]{spaces-descent}
\externaldocument[spaces-perfect-]{spaces-perfect}
\externaldocument[spaces-more-morphisms-]{spaces-more-morphisms}
\externaldocument[spaces-flat-]{spaces-flat}
\externaldocument[spaces-groupoids-]{spaces-groupoids}
\externaldocument[spaces-more-groupoids-]{spaces-more-groupoids}
\externaldocument[bootstrap-]{bootstrap}
\externaldocument[spaces-pushouts-]{spaces-pushouts}
\externaldocument[spaces-chow-]{spaces-chow}
\externaldocument[groupoids-quotients-]{groupoids-quotients}
\externaldocument[spaces-more-cohomology-]{spaces-more-cohomology}
\externaldocument[spaces-simplicial-]{spaces-simplicial}
\externaldocument[spaces-duality-]{spaces-duality}
\externaldocument[formal-spaces-]{formal-spaces}
\externaldocument[restricted-]{restricted}
\externaldocument[spaces-resolve-]{spaces-resolve}
\externaldocument[formal-defos-]{formal-defos}
\externaldocument[defos-]{defos}
\externaldocument[cotangent-]{cotangent}
\externaldocument[examples-defos-]{examples-defos}
\externaldocument[algebraic-]{algebraic}
\externaldocument[examples-stacks-]{examples-stacks}
\externaldocument[stacks-sheaves-]{stacks-sheaves}
\externaldocument[criteria-]{criteria}
\externaldocument[artin-]{artin}
\externaldocument[quot-]{quot}
\externaldocument[stacks-properties-]{stacks-properties}
\externaldocument[stacks-morphisms-]{stacks-morphisms}
\externaldocument[stacks-limits-]{stacks-limits}
\externaldocument[stacks-cohomology-]{stacks-cohomology}
\externaldocument[stacks-perfect-]{stacks-perfect}
\externaldocument[stacks-introduction-]{stacks-introduction}
\externaldocument[stacks-more-morphisms-]{stacks-more-morphisms}
\externaldocument[stacks-geometry-]{stacks-geometry}
\externaldocument[moduli-]{moduli}
\externaldocument[moduli-curves-]{moduli-curves}
\externaldocument[examples-]{examples}
\externaldocument[exercises-]{exercises}
\externaldocument[guide-]{guide}
\externaldocument[desirables-]{desirables}
\externaldocument[coding-]{coding}
\externaldocument[obsolete-]{obsolete}
\externaldocument[fdl-]{fdl}
\externaldocument[index-]{index}

% Theorem environments.
%
\theoremstyle{plain}
\newtheorem{theorem}[subsection]{Theorem}
\newtheorem{proposition}[subsection]{Proposition}
\newtheorem{lemma}[subsection]{Lemma}

\theoremstyle{definition}
\newtheorem{definition}[subsection]{Definition}
\newtheorem{example}[subsection]{Example}
\newtheorem{exercise}[subsection]{Exercise}
\newtheorem{situation}[subsection]{Situation}

\theoremstyle{remark}
\newtheorem{remark}[subsection]{Remark}
\newtheorem{remarks}[subsection]{Remarks}

\numberwithin{equation}{subsection}

% Macros
%
\def\lim{\mathop{\mathrm{lim}}\nolimits}
\def\colim{\mathop{\mathrm{colim}}\nolimits}
\def\Spec{\mathop{\mathrm{Spec}}}
\def\Hom{\mathop{\mathrm{Hom}}\nolimits}
\def\Ext{\mathop{\mathrm{Ext}}\nolimits}
\def\SheafHom{\mathop{\mathcal{H}\!\mathit{om}}\nolimits}
\def\SheafExt{\mathop{\mathcal{E}\!\mathit{xt}}\nolimits}
\def\Sch{\mathit{Sch}}
\def\Mor{\mathop{\mathrm{Mor}}\nolimits}
\def\Ob{\mathop{\mathrm{Ob}}\nolimits}
\def\Sh{\mathop{\mathit{Sh}}\nolimits}
\def\NL{\mathop{N\!L}\nolimits}
\def\CH{\mathop{\mathrm{CH}}\nolimits}
\def\proetale{{pro\text{-}\acute{e}tale}}
\def\etale{{\acute{e}tale}}
\def\QCoh{\mathit{QCoh}}
\def\Ker{\mathop{\mathrm{Ker}}}
\def\Im{\mathop{\mathrm{Im}}}
\def\Coker{\mathop{\mathrm{Coker}}}
\def\Coim{\mathop{\mathrm{Coim}}}

% Boxtimes
%
\DeclareMathSymbol{\boxtimes}{\mathbin}{AMSa}{"02}

%
% Macros for moduli stacks/spaces
%
\def\QCohstack{\mathcal{QC}\!\mathit{oh}}
\def\Cohstack{\mathcal{C}\!\mathit{oh}}
\def\Spacesstack{\mathcal{S}\!\mathit{paces}}
\def\Quotfunctor{\mathrm{Quot}}
\def\Hilbfunctor{\mathrm{Hilb}}
\def\Curvesstack{\mathcal{C}\!\mathit{urves}}
\def\Polarizedstack{\mathcal{P}\!\mathit{olarized}}
\def\Complexesstack{\mathcal{C}\!\mathit{omplexes}}
% \Pic is the operator that assigns to X its picard group, usage \Pic(X)
% \Picardstack_{X/B} denotes the Picard stack of X over B
% \Picardfunctor_{X/B} denotes the Picard functor of X over B
\def\Pic{\mathop{\mathrm{Pic}}\nolimits}
\def\Picardstack{\mathcal{P}\!\mathit{ic}}
\def\Picardfunctor{\mathrm{Pic}}
\def\Deformationcategory{\mathcal{D}\!\mathit{ef}}


\begin{document}

\title{Categories}
\maketitle

\phantomsection
\label{section-phantom}
\hfill
\href{http://github.com/danimalabares/stack}{github.com/danimalabares/stack}

\tableofcontents

\section{Definitions}
\label{section-definition-categories}

\noindent
We recall the definitions, partly to fix notation.

\begin{definition}
\label{definition-category}
A {\it category} $\mathcal{C}$ consists of the following data:
\begin{enumerate}
\item A set of objects $\Ob(\mathcal{C})$.
\item For each pair $x, y \in \Ob(\mathcal{C})$ a set of morphisms
$\Mor_\mathcal{C}(x, y)$.
\item For each triple $x, y, z\in \Ob(\mathcal{C})$ a composition
map $ \Mor_\mathcal{C}(y, z) \times \Mor_\mathcal{C}(x, y)
\to \Mor_\mathcal{C}(x, z) $, denoted $(\phi, \psi) \mapsto
\phi \circ \psi$.
\end{enumerate}
These data are to satisfy the following rules:
\begin{enumerate}
\item For every element $x\in \Ob(\mathcal{C})$ there exists a
morphism $\text{id}_x\in \Mor_\mathcal{C}(x, x)$ such that
$\text{id}_x \circ \phi = \phi$ and $\psi \circ \text{id}_x = \psi $ whenever
these compositions make sense.
\item Composition is associative, i.e., $(\phi \circ \psi) \circ \chi =
\phi \circ ( \psi \circ \chi)$ whenever these compositions make sense.
\end{enumerate}
\end{definition}

\begin{definition}
\label{definition-functor}
A {\it functor} $F : \mathcal{A} \to \mathcal{B}$
between two categories $\mathcal{A}, \mathcal{B}$ is given by the
following data:
\begin{enumerate}
\item A map $F : \Ob(\mathcal{A}) \to \Ob(\mathcal{B})$.
\item For every $x, y \in \Ob(\mathcal{A})$ a map
$F : \Mor_\mathcal{A}(x, y) \to \Mor_\mathcal{B}(F(x), F(y))$,
denoted $\phi \mapsto F(\phi)$.
\end{enumerate}
These data should be compatible with composition and identity morphisms
in the following manner: $F(\phi \circ \psi) =
F(\phi) \circ F(\psi)$ for a composable pair $(\phi, \psi)$ of
morphisms of $\mathcal{A}$ and $F(\text{id}_x) = \text{id}_{F(x)}$.
\end{definition}

\section{Monomorphisms}
\label{section-monomorphisms-and-epimorphisms}

\begin{definition}
\label{definition-mono-epi}
Let $\mathcal{C}$ be a category and let $f : X \to Y$ be
a morphism of $\mathcal{C}$.
\begin{enumerate}
\item We say that $f$ is a {\it monomorphism} if for every object
$W$ and every pair of morphisms $a, b : W \to X$ such that
$f \circ a = f \circ b$ we have $a = b$.
\item We say that $f$ is an {\it epimorphism} if for every object
$W$ and every pair of morphisms $a, b : Y \to W$ such that
$a \circ f = b \circ f$ we have $a = b$.
\end{enumerate}
\end{definition}

\begin{definition}
\label{definition-presheaves-injective-surjective}
Let $\mathcal{C}$ be a category, and let $\varphi : \mathcal{F}
\to \mathcal{G}$ be a map of presheaves of sets.
\begin{enumerate}
\item We say that $\varphi$ is {\it injective} if for every object
$U$ of $\mathcal{C}$ the map $\varphi_U : \mathcal{F}(U)
\to \mathcal{G}(U)$ is injective.
\item We say that $\varphi$ is {\it surjective} if for every object
$U$ of $\mathcal{C}$ the map $\varphi_U : \mathcal{F}(U)
\to \mathcal{G}(U)$ is surjective.
\end{enumerate}
\end{definition}

\begin{lemma}
\label{lemma-mono-epi}
The injective (resp.\ surjective) maps defined above
are exactly the monomorphisms (resp.\ epimorphisms) of
$\textit{PSh}(\mathcal{C})$. A map is an isomorphism
if and only if it is both injective and surjective.
\end{lemma}

\section{Presheaves}
\label{section-presheaves}

\begin{definition}
\label{definition-presheaves-sets}
A {\it presheaf of sets} on $\mathcal{C}$ is a contravariant
functor from $\mathcal{C}$ to $\textit{Sets}$. {\it Morphisms
of presheaves} are natural transformations of functors.
The category
of presheaves of sets is denoted $\textit{PSh}(\mathcal{C})$
or $\hat{\mathcal{C}}$.
\end{definition}

\section{Yoneda lemma}
\label{section-Yoneda-lemma}

The Yoneda lemma says that the sections over $a$
of a presheaf $X$ can be described completely
as natural transformations between the 
presheaf $\Hom(-,a)$ and $X$.

\begin{definition}
\label{definition-Yoneda-embedding}
\begin{reference}
\cite[Definition 1.1.3]{Cisinsky}
\end{reference}
Let $A$ be a category.
The {\it Yoneda embedding} is the functor
$$
h:A \to \hat{A}
$$
whose value at an object $a$ of $A$ is the presheaf
$$
h_a=\Hom_A(-,a).
$$
\end{definition}

In other words, the evaluation of the presheaf
$h_a$ at an object $c$ of $A$ is the set of
maps from $c$ to $a$.

\begin{theorem}[Yoneda lemma]
\label{theorem-Yoneda-lemma}
\begin{reference}
\cite[Theorem 1.1.4]{Cisinsky}
\end{reference}
For any presheaf $X$ over $A$, there is a natural bijection
(in $\text{Sets}$ I think!!)
\begin{align*}
\Hom_{\widehat{A}}(h_a,X) &\xrightarrow{\simeq} X_a  \\
(h_a \xrightarrow{u}X) &\longmapsto u_a(1_a)
\end{align*}

\end{theorem}

\section{Internal Hom}
\label{section-internal-hom}

\noindent
{\bf Upshot.} Internal Hom is when the
Hom set of two objects in some category
is in also an object of the category.
Down-to-earth, that for two sheaves $\mathcal{F},\mathcal{G}$,
$U\mapsto \Hom(\mathcal{F}(U),\mathcal{G}(U)$ 
is also a sheaf, called $\SheafHom$.

\medskip\noindent
I start with Stacks Project approach.

Let $(X, \mathcal{O}_X)$ be a ringed space.
Let $\mathcal{F}$, $\mathcal{G}$ be $\mathcal{O}_X$-modules.
Consider the rule
$$
U \longmapsto \Hom_{\mathcal{O}_X|_U}(\mathcal{F}|_U, \mathcal{G}|_U).
$$
It follows from the discussion in Sheaves, Section
\ref{sheaves-section-glueing-sheaves} that this is a sheaf of
abelian groups. In addition, given an element
$\varphi \in \Hom_{\mathcal{O}_X|_U}(\mathcal{F}|_U, \mathcal{G}|_U)$
and a section $f \in \mathcal{O}_X(U)$ then we can define
$f\varphi \in \Hom_{\mathcal{O}_X|_U}(\mathcal{F}|_U, \mathcal{G}|_U)$
by either precomposing with multiplication by $f$ on $\mathcal{F}|_U$
or postcomposing with multiplication by $f$ on $\mathcal{G}|_U$ (it gives
the same result). Hence we in fact get a sheaf of $\mathcal{O}_X$-modules.
We will denote this sheaf
$\SheafHom_{\mathcal{O}_X}(\mathcal{F}, \mathcal{G})$.
There is a canonical ``evaluation'' morphism
$$
\mathcal{F}
\otimes_{\mathcal{O}_X}
\SheafHom_{\mathcal{O}_X}(\mathcal{F}, \mathcal{G})
\longrightarrow
\mathcal{G}.
$$
For every $x \in X$ there is also a canonical morphism
$$
\SheafHom_{\mathcal{O}_X}(\mathcal{F}, \mathcal{G})_x
\to
\Hom_{\mathcal{O}_{X, x}}(\mathcal{F}_x, \mathcal{G}_x)
$$
which is rarely an isomorphism.

\medskip\noindent
{\bf Cartesian closed category} In the category of sets there is a bijection
$\Hom(X\times Y, Z)\cong \Hom(X, \text{Hom}(Y, Z))$ that depends
naturally on $X$, $Y$ and $Z$. The notions related to this bijection are
“Cartesian closed category”, “currying” and “internal Hom”.

\begin{definition}
\label{definition-Cartesian-closed-category}
A category $\mathcal{C}$ is {\it Cartesian closed} if:
\begin{enumerate}
\item $\mathcal{C}$ has all finite products 
(Caveat: some require that $\mathcal{C}$ has all finite limits)
\item For any object $Y$ the functor $- \times Y$ has a right adjoint,
which we will denote by $\text{Map}(Y,-)$ or by $-^Y$ .
\end{enumerate}
\end{definition}

\begin{remark}
\label{remark-currying}
By section 3 \href{https://ncatlab.org/nlab/show/internal+hom }{here}, the
second property above implies that we get a functor 
$\text{Map}(-,-) :\mathcal{C}^{\text{op}} \times \mathcal{C} \to \mathcal{C}$,
and moreover we get natural isomorphisms 
$\Hom(X, \text{Map}(Y, Z)) \cong \Hom(X \times Y, Z)$
and $\text{Map}(X, \text{Map}(Y, Z))\cong \text{Map}(X \times Y, Z)$.
\end{remark}

\section{Product}
\label{section-product}

\begin{definition}
\label{definition-product}
A {\it product} in a category $\mathcal{C}$ is
$$
\xymatrix{
P\ar[d]_b \ar[r]^a&A\\
B
}
$$
such that for every other
$$
\xymatrix{
P'\ar[d]_{b'} \ar[r]^{a'}&A\\
B
}
$$
there exists a unique map $p:P'\to P$ such that
$ap=a'$ and  $bp=b'$.

\end{definition}

For completeness here's Stacks Project formulation:

\begin{definition}
\label{definition-products}
Let $x, y\in \Ob(\mathcal{C})$.
A {\it product} of $x$ and $y$ is
an object $x \times y \in \Ob(\mathcal{C})$
together with morphisms
$p\in \Mor_{\mathcal C}(x \times y, x)$ and
$q\in\Mor_{\mathcal C}(x \times y, y)$ such
that the following universal property holds: for
any $w\in \Ob(\mathcal{C})$ and morphisms
$\alpha \in \Mor_{\mathcal C}(w, x)$ and
$\beta \in \Mor_\mathcal{C}(w, y)$
there is a unique
$\gamma\in \Mor_{\mathcal C}(w, x \times y)$ making
the diagram
$$
\xymatrix{
w \ar[rrrd]^\beta \ar@{-->}[rrd]_\gamma \ar[rrdd]_\alpha & & \\
& & x \times y \ar[d]_p \ar[r]_q & y \\
& & x &
}
$$
commute.
\end{definition}

And some nice piece of information and a definition also from Stacks Project.

\medskip\noindent
If a product exists it is unique up to unique
isomorphism. This follows from the Yoneda lemma as
the definition requires $x \times y$ to be an object
of $\mathcal{C}$ such that
$$
h_{x \times y}(w) = h_x(w) \times h_y(w)
$$
functorially in $w$. In other words the product $x \times y$
is an object representing the functor
$w \mapsto h_x(w) \times h_y(w)$.

\begin{definition}
\label{definition-has-products-of-pairs}
We say the category $\mathcal{C}$ {\it has products of pairs
of objects} if a product $x \times y$
exists for any $x, y \in \Ob(\mathcal{C})$.
\end{definition}

\section{Coproducts of pairs}
\label{section-coproducts-pairs}

\begin{definition}
\label{definition-coproducts}
Let $x, y \in \Ob(\mathcal{C})$.
A {\it coproduct}, or {\it amalgamated sum} of $x$ and $y$ is
an object $x \amalg y \in \Ob(\mathcal{C})$
together with morphisms
$i \in \Mor_{\mathcal C}(x, x \amalg y)$ and
$j \in \Mor_{\mathcal C}(y, x \amalg y)$ such
that the following universal property holds: for
any $w \in \Ob(\mathcal{C})$ and morphisms
$\alpha \in \Mor_{\mathcal C}(x, w)$ and
$\beta \in \Mor_\mathcal{C}(y, w)$
there is a unique
$\gamma \in \Mor_{\mathcal C}(x \amalg y, w)$ making
the diagram
$$
\xymatrix{
& y \ar[d]^j \ar[rrdd]^\beta \\
x \ar[r]^i \ar[rrrd]_\alpha & x \amalg y \ar@{-->}[rrd]^\gamma \\
& & & w
}
$$
commute.
\end{definition}

\noindent
If a coproduct exists it is unique up to unique
isomorphism. This follows from the Yoneda lemma (applied to the
opposite category) as
the definition requires $x \amalg y$ to be an object
of $\mathcal{C}$ such that
$$
\Mor_\mathcal{C}(x \amalg y, w) =
\Mor_\mathcal{C}(x, w) \times \Mor_\mathcal{C}(y, w)
$$
functorially in $w$.

\begin{definition}
\label{definition-has-coproducts-of-pairs}
We say the category $\mathcal{C}$ {\it has coproducts of pairs
of objects} if a coproduct $x \amalg y$
exists for any $x, y \in \Ob(\mathcal{C})$.
\end{definition}
\section{Fibre products}
\label{section-fibre-products}

\begin{definition}
\label{definition-fibre-products}
Let $x, y, z\in \Ob(\mathcal{C})$,
$f\in \Mor_\mathcal{C}(x, y)$
and $g\in \Mor_{\mathcal C}(z, y)$.
A {\it fibre product} of $f$ and $g$ is
an object $x \times_y z\in \Ob(\mathcal{C})$
together with morphisms
$p \in \Mor_{\mathcal C}(x \times_y z, x)$ and
$q \in \Mor_{\mathcal C}(x \times_y z, z)$ making the diagram
$$
\xymatrix{
x \times_y z \ar[r]_q \ar[d]_p & z \ar[d]^g \\
x \ar[r]^f & y
}
$$
commute, and such that the following universal property holds: for
any $w\in \Ob(\mathcal{C})$ and morphisms
$\alpha \in \Mor_{\mathcal C}(w, x)$ and
$\beta \in \Mor_\mathcal{C}(w, z)$ with
$f \circ \alpha = g \circ \beta$
there is a unique
$\gamma \in \Mor_{\mathcal C}(w, x \times_y z)$ making
the diagram
$$
\xymatrix{
w \ar[rrrd]^\beta \ar@{-->}[rrd]_\gamma \ar[rrdd]_\alpha & & \\
& & x \times_y z \ar[d]^p \ar[r]_q & z \ar[d]^g \\
& & x \ar[r]^f & y
}
$$
commute.
\end{definition}

\noindent
If a fibre product exists it is unique up to unique
isomorphism. This follows from the Yoneda lemma as
the definition requires $x \times_y z$ to be an object
of $\mathcal{C}$ such that
$$
h_{x \times_y z}(w) = h_x(w) \times_{h_y(w)} h_z(w)
$$
functorially in $w$. In other words the fibre product $x \times_y z$
is an object representing the functor
$w \mapsto h_x(w) \times_{h_y(w)} h_z(w)$.

\begin{definition}
\label{definition-cartesian}
We say a commutative diagram
$$
\xymatrix{
w \ar[r] \ar[d] &
z \ar[d] \\
x \ar[r] &
y
}
$$
in a category is {\it cartesian} if $w$ and the morphisms $w \to x$ and
$w \to z$ form a fibre product of the morphisms $x \to y$ and $z \to y$.
\end{definition}

\begin{definition}
\label{definition-has-fibre-products}
We say the category $\mathcal{C}$ {\it has fibre products} if
the fibre product exists for any $f\in \Mor_{\mathcal C}(x, y)$
and $g\in \Mor_{\mathcal C}(z, y)$.
\end{definition}

\begin{definition}
\label{definition-representable-morphism}
A morphism $f : x \to y$ of a category $\mathcal{C}$ is said to be
{\it representable} if for every morphism $z \to y$
in $\mathcal{C}$ the fibre product $x \times_y z$ exists.
\end{definition}

\begin{lemma}
\label{lemma-composition-representable}
Let $\mathcal{C}$ be a category.
Let $f : x \to y$, and $g : y \to z$ be representable.
Then $g \circ f : x \to z$ is representable.
\end{lemma}

\begin{proof}
Let $t \in \Ob(\mathcal C)$ and $ \varphi \in \Mor_{\mathcal C}(t,z) $.
As $g$ and $f$ are representable, we obtain commutative diagrams
$$
\xymatrix{
y \times_z t \ar[r]_q \ar[d]_p &
t \ar[d]^{\varphi} \\
y \ar[r]^{g} & z
}
\quad\quad
\xymatrix{
x \times_y (y\! \times_z\! t) \ar[r]_{q'} \ar[d]_{p'} &
y \times_z t \ar[d]^p \\
x \ar[r]^f & y
}
$$
with the universal property of Definition \ref{definition-fibre-products}.
We claim that $x \times_z t = x \times_y (y \times_z t)$
with morphisms $q \circ q' : x \times_z t \to t$ and $p' : x \times_z t \to x$
is a fibre product. First, it follows from the commutativity of the
diagrams above that $\varphi \circ q \circ q' = g \circ f \circ p'$.
To verify the universal property, let
$w \in \Ob(\mathcal C)$ and suppose $\alpha : w \to x$ and
$\beta : w \to y$ are morphisms with
$\varphi \circ \beta = g \circ f \circ \alpha$.
By definition of the fibre product, there
are unique morphisms $\delta$ and $\gamma$ such that
$$
\xymatrix{
w \ar[rrrd]^\beta \ar@{-->}[rrd]_\delta \ar[rrdd]_{f\circ\alpha} & & \\
& & y \times_z t \ar[d]_p \ar[r]_q & t \ar[d]^{\varphi} \\
& & y \ar[r]^{g} & z
}
$$
and
$$
\xymatrix{
w \ar[rrrd]^\delta \ar@{-->}[rrd]_\gamma \ar[rrdd]_{\alpha} & & \\
& & x \times_y (y\!\times_z\! t) \ar[d]_{p'} \ar[r]_{q'} &
y \times_z t \ar[d]^{p} \\
& & x \ar[r]^{f} & y
}
$$
commute. Then, $\gamma$ makes the diagram
$$
\xymatrix{
w \ar[rrrd]^\beta \ar@{-->}[rrd]_\gamma \ar[rrdd]_{\alpha} & & \\
& & x \times_z t \ar[d]_{p'} \ar[r]_{q\circ q'} & t \ar[d]^{\varphi} \\
& & x \ar[r]^{g\circ f} & z
}
$$
commute. To show its uniqueness, let $\gamma'$ verify $ q\circ q'\circ\gamma' =
\beta $ and $ p'\circ \gamma' = \alpha $. Because $\gamma$ is unique, we just
need to prove that $ q'\circ\gamma' = \delta $ and $ p'\circ\gamma' = \alpha $
to conclude. We supposed the second equality. For the first one, we also need
to use the uniqueness of delta. Notice that $\delta$ is the only morphism
verifying $ q\circ\delta = \beta $ and $ p\circ\delta = f\circ\alpha $. We
already supposed that $ q\circ (q'\circ\gamma') = \beta $. Furthermore, by
definition of the fibre product, we know that $ f\circ p' = p\circ q' $.
Therefore:
$$
p\circ (q'\circ\gamma') =
(p\circ q')\circ\gamma' =
(f\circ p')\circ\gamma' =
f\circ (p'\circ\gamma') = f\circ\alpha.
$$
Then $ q'\circ\gamma' = \delta $, which concludes the proof.
\end{proof}

\begin{lemma}
\label{lemma-base-change-representable}
Let $\mathcal{C}$ be a category.
Let $f : x \to y$ be representable.
Let $y' \to y$ be a morphism of $\mathcal{C}$.
Then the morphism $x' := x \times_y y' \to y'$ is representable also.
\end{lemma}

\begin{proof}
Let $z \to y'$ be a morphism. The fibre product
$x' \times_{y'} z$ is supposed to represent the
functor
\begin{eqnarray*}
w & \mapsto & h_{x'}(w)\times_{h_{y'}(w)} h_z(w) \\
& = & (h_x(w) \times_{h_y(w)} h_{y'}(w)) \times_{h_{y'}(w)} h_z(w) \\
& = & h_x(w) \times_{h_y(w)} h_z(w)
\end{eqnarray*}
which is representable by assumption.
\end{proof}

\section{Examples of fibre products}
\label{section-example-fibre-products}

\noindent
In this section we list examples of fibre products and
we describe them.

\medskip\noindent
As a really trivial first example we observe
that the category of sets has fibre products and hence every
morphism is representable. Namely, if $f : X \to Y$
and $g : Z \to Y$ are maps of sets then we define
$X \times_Y Z$ as the subset of $X \times Z$ consisting
of pairs $(x, z)$ such that $f(x) = g(z)$. The morphisms
$p : X \times_Y Z \to X$ and $q : X \times_Y Z \to Z$ are
the projection maps $(x, z) \mapsto x$, and $(x, z) \mapsto z$.
Finally, if $\alpha : W \to X$ and $\beta : W \to Z$
are morphisms such that $f \circ \alpha = g \circ \beta$
then the map $W \to X \times Z$, $w\mapsto (\alpha(w), \beta(w))$
obviously ends up in $X \times_Y Z$ as desired.

\medskip\noindent
In many categories whose objects are sets endowed with certain types of
algebraic structures the fibre product of the underlying sets also
provides the fibre product in the category. For example, suppose
that $X$, $Y$ and $Z$ above are groups and that $f$, $g$ are
homomorphisms of groups. Then the set-theoretic fibre product
$X \times_Y Z$ inherits the structure of a group, simply by
defining the product of two pairs by the formula
$(x, z) \cdot (x', z') = (xx', zz')$. Here we list those categories
for which a similar reasoning works.
\begin{enumerate}
\item The category $\textit{Groups}$ of groups.
\item The category $G\textit{-Sets}$ of sets
endowed with a left $G$-action for some fixed group $G$.
\item The category of rings.
\item The category of $R$-modules given a ring $R$.
\end{enumerate}



\section{Limits and colimits}
\label{section-limits-and-colimits}

The definition of product is actually a particular case of a limit.

\medskip\noindent
Let $\mathcal{C}$ be a category. A {\it diagram} in $\mathcal{C}$ is
simply a functor $M : \mathcal{I} \to \mathcal{C}$. We say that
$\mathcal{I}$ is the {\it index category} or that $M$ is an
$\mathcal{I}$-diagram. We will use the notation $M_i$ to denote the
image of the object
$i$ of $\mathcal{I}$. Hence for $\phi : i \to i'$ a morphism
in $\mathcal{I}$ we have $M(\phi) : M_i \to M_{i'}$.

\begin{definition}
\label{definition-limit}
A {\it limit} of the $\mathcal{I}$-diagram $M$ in the category
$\mathcal{C}$ is given by an object $\lim_\mathcal{I} M$ in $\mathcal{C}$
together with morphisms $p_i : \lim_\mathcal{I} M \to M_i$ such that
\begin{enumerate}
\item for $\phi : i \to i'$ a morphism
in $\mathcal{I}$ we have $p_{i'} = M(\phi) \circ p_i$, and
\item for any object $W$ in $\mathcal{C}$ and any family of
morphisms $q_i : W \to M_i$ (indexed by $i \in \Ob(\mathcal{I})$)
such that for all $\phi : i \to i'$
in $\mathcal{I}$ we have $q_{i'} = M(\phi) \circ q_i$ there
exists a unique morphism $q : W \to \lim_\mathcal{I} M$ such that
$q_i = p_i \circ q$ for every object $i$ of $\mathcal{I}$.
\end{enumerate}
\end{definition}

\noindent
Limits $(\lim_\mathcal{I} M, (p_i)_{i\in \Ob(\mathcal{I})})$ are
(if they exist)
unique up to unique isomorphism by the uniqueness requirement
in the definition. Products of pairs, fibre products, and equalizers are
examples of limits. The limit over the empty diagram is a final object
of $\mathcal{C}$.
In the category of sets all limits exist.
The dual notion is that of colimits.

\begin{definition}
\label{definition-colimit}
A {\it colimit} of the $\mathcal{I}$-diagram $M$ in the category
$\mathcal{C}$ is given by an object $\colim_\mathcal{I} M$ in $\mathcal{C}$
together with morphisms $s_i : M_i \to \colim_\mathcal{I} M$ such that
\begin{enumerate}
\item for $\phi : i \to i'$ a morphism
in $\mathcal{I}$ we have $s_i = s_{i'} \circ M(\phi)$, and
\item for any object $W$ in $\mathcal{C}$ and any family of
morphisms $t_i : M_i \to W$ (indexed by $i \in \Ob(\mathcal{I})$)
such that for all $\phi : i \to i'$
in $\mathcal{I}$ we have $t_i = t_{i'} \circ M(\phi)$ there
exists a unique morphism $t : \colim_\mathcal{I} M \to W$ such that
$t_i = t \circ s_i$ for every object $i$ of $\mathcal{I}$.
\end{enumerate}
\end{definition}

\begin{definition}
\label{definition-product-as-a-limit}
Suppose that $I$ is a set, and suppose given for every $i \in I$ an
object $M_i$ of the category $\mathcal{C}$. A {\it product}
$\prod_{i\in I} M_i$ is by definition $\lim_\mathcal{I} M$
(if it exists)
where $\mathcal{I}$ is the category having only identities as
morphisms and having the elements of $I$ as objects.
\end{definition}

\noindent
An important special case is where $I = \emptyset$ in which case the
product is a final object of the category.
The morphisms $p_i : \prod M_i \to M_i$ are called the
{\it projection morphisms}.

\begin{definition}
\label{definition-coproduct}
Suppose that $I$ is a set, and suppose given for every $i \in I$ an
object $M_i$ of the category $\mathcal{C}$. A {\it coproduct}
$\coprod_{i\in I} M_i$ is by definition $\colim_\mathcal{I} M$
(if it exists) where $\mathcal{I}$ is the category having only
identities as morphisms and having the elements of $I$ as objects.
\end{definition}

\noindent
An important special case is where $I = \emptyset$ in which case the
coproduct is an initial object of the category.
Note that the coproduct comes equipped with morphisms
$M_i \to \coprod M_i$. These are sometimes called the
{\it coprojections}.



\section{Simplicial sets}
\label{section-simplicial-sets}

\begin{definition}
\label{definition-simplex}
A {\it simplex} is a non-empty finite ordered set
$$
[n]=\{0<1<\ldots<n\}, \quad n=0,1,\ldots
$$
Simplexes form the category $\Delta$ along with order preserving
maps of sets.
\end{definition}

\begin{definition}
\label{definition-simplicial-set}
A {\it simplicial set} is a presheaf on $\Delta$, i.e.
an element of $\mathsf{Fun}(\Delta^{\text{op}},\text{Sets})$.
\end{definition}




\bibliography{my} \bibliographystyle{amsalpha}



\end{document}

