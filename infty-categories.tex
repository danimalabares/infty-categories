\input{preamble}

\begin{document}

\title{Infty-categories}
\maketitle

\phantomsection
\label{section-phantom}
\hfill
\href{http://github.com/danimalabares/infty-categories}
{github.com/danimalabares/infty-categories}

\tableofcontents

\section{Other exercises}
\label{section-other-exercises}

\begin{definition}
\label{definition-discrete-fibration}
\begin{reference}
\url{https://ncatlab.org/nlab/show/discrete+fibration} 
\end{reference}
A functor $F:C \to B$ is a {\it discrete fibration} if for every object
$c$ in $C$ and every morphism of the form $g: b \to F(c)$ in $B$ 
there is a unique morphism $h:d \to c$ in $C$ 
such that $F(h)=g$.
\end{definition}

\begin{exercise}
\label{exercise-discrete-fibrations-over-category-are-presheaves}
Prove that discrete fibrations over a category $C$ correspond to presheaves over
$C$.
\end{exercise}

\begin{proof}
First suppose that we are given a presheaf $X$ on $C$. Define a discrete
fibration $F:C/X \to C$ by $(a,s)\mapsto a$ on objects and
mapping a morphism $f:(a,s)\to (b,t)$ in $C/X$ to the
corresponding morphism $a \to b$ in $C$.
To show $F$ is a discrete fibration 
let $g:b\to a$ be a morphism in $C$. 
Consider $g^*=X(g):X_a\to X_b$,
and the section $g^*s$ of $X_b$.
Then the
morphism $h:(b,g^*s)\to (a,s)$ is the only one mapping to $g$ under $h$.

For the converse let $F:B \to C$ be a discrete fibration over $C$.
To define a presheaf $X:C^{\text{op}}\to \text{Sets}$ let $c \in \Ob C$.
We assign the set (for now I won't justify why this is a set)
of objects in $B$ mapped to $c$ under $F$.
To define the correspondence on morphisms, 
consider a map $f:c \to d$ in $C^{\text{op}}$.
In other words, we have a map in  $C$ of the form $f^{\text{op}}:d \to c$.
Then to any object in $b$ such that $F(b)=c$,
by definition of discrete fibration,
we have a unique morphism of $B$ of the form
$h:r \to b$ such that $F(h)=f^{\text{op}}$.
In particular this means that $F(r)=d$.
This gives a function from $X(c)$ to $X(d)$.
This situation is described in the following diagram:

$$
\xymatrix{
X(d)\ni&r \ar@{.>}[d]_{\exists!h}\ar@{|->}[r]& d=F(r)
\ar[d]^{f^{\text{op}}=F(h)}\\
X(c)\ni& b\ar@{|->}[r]_F&c=F(b)
}
$$




To check functoriality of $X$ defined in the previous paragraph
suppose that $f:c\to d$ and $g:d\to e$ are two morphisms in $C^{\text{op}}$.
Like before, we have maps $f^{\text{op}}:d\to c$ and $g^{\text{op}}:e \to d$.
$$
\xymatrix{
X(e)\ni& q\ar@{.>}[d]_{\exists !j}\ar@{|->}[r]& e=F(q)
\ar[d]^{g^{\text{op}}=F(j)}\\
X(d)\ni&r \ar@{.>}[d]_{\exists!h}\ar@{|->}[r]& d=F(r)
\ar[d]^{f^{\text{op}}=F(h)}\\
X(c)\ni& b\ar@{|->}[r]_F&c=F(b)
}
$$
on the other hand, $gf:c \to e$ gives by the
same construction a unique map $k:\hat{q}  \to b$
such that $F(k)=f^{\text{op}}g^{\text{op}}$.
To check that $\hat{q}=q$, observe that
by functoriality of $F$, we have
$F(hj)=F(h)F(j)=f^{\text{op}}g^{\text{op}}=F(k)$.
By uniqueness of $k$, we conclude that
$k=hj$ and thus $q=\hat{q}$.
\end{proof}

\section{Exercises of Rune, Chapter 1}
\label{section-exercises-of-Rune-Chapter-1}

Here's my progress so far on the exercises in \cite{Rune}, Chapter 1.

\begin{exercise}[Observation 1.4.7]
\label{exercise-simplicial-sets-category-has-internal-Hom}
Show that the simplicial sets category  $\mathsf{Set}_\Delta$ has internal 
Hom $S^T$ for simplicial sets $S$ and $T$, given by
$$
(S^T)_n:=\Hom_{\mathsf{Set}_\Delta}(T\times\Delta^n,S)
$$
\end{exercise}

\begin{proof}
We need to show, that $S^T$ is the internal Hom in the category
$\mathsf{Set}_\Delta$. Different notations for the internal Hom
are $\text{Map}(-,-)$, $\underline{\Hom}(-,-)$.
It must be right adjoint to 
the functor $U \times -$.
That is,
$$
\Hom_{\text{Set}_\Delta}(U \times S,T)\cong\Hom_{\mathsf{Set}_\Delta}(U,S^T)
$$
(I think) I understand the statement correctly
but I don't understand how to apply
\cite[Theorem 1.1.10 (Kan)]{Cisinsky} nor \cite[Remark 1.1.11]{Cisinsky}
to prove it.


\end{proof}

\begin{exercise}[1.1]
\label{exercise-relation-is-equivalence-relation}
If $S$ is a Kan complex, then the relation defining
$\pi_0S$ is an equivalence relation.
\end{exercise}

\begin{proof}
\begin{enumerate}
\item (Reflexivity.) Let $a \in S_0$. Consider the composition
$$
\xymatrixrowsep{0.5em}\xymatrix{
[0]\ar[r]^{d_0}& [1]\ar[r]^{f_0}& [0]\\
0\ar@{|->}[r]& 1\ar@{|->}[r]&0
}
$$
since this gives the identity we must have
$$
\xymatrixrowsep{0.5em}
\xymatrix{
S_0\ar[r]^{S(f_0)}& S_1\ar[r]^{S(d_0)}& [0]\\
a\ar@{|->}[r]& S(f_0)(a)\ar@{|->}[r]&a
}
$$
but we can replace $d_0$ by $d_1$ and we'd still get the identity,
so that $S(d_1)$ also maps $S(f_0)(a)$ to $a$. In other words, 
for any  $a \in S_0$ the 1-simplex $S(f_0)(a)$ is the desired one.
\item (Symmetry.) Let $a,b \in S_0$…
\end{enumerate}
\end{proof}

Rather informally, I understand a category $C$ to be
an {\it enriched category over $D$} if for any objects
$c,d$ in $C$, $\Hom(c,d)$ is an object of $D$.
Compositions of morphisms exist and are associative,
and there is an identity morphism for every object $c$ in $C$.
(See \url{https://ncatlab.org/nlab/show/enriched+category} 
for a formal definition.)

\begin{exercise}[1.2]
\label{exercise-enriched-simplicial-set-categories-are-subcategory-of-Fun}
Show that $\mathsf{Cat}_\Delta$ can be described as the full
subcategory of $\text{Fun}(\Delta^{\text{op}},\mathsf{Cat})$
containing the functors whose simplicial sets of objects
are constant.
\end{exercise}

\begin{remark}
\label{remark-induced-functor-on-objects}
The phrase ``simplicial sets of objects are constants''
means the following. Consider the functor
$\mathsf{Cat} \to \mathsf{Set}$ that maps a category to its set
of objects (I suppose we may take $\mathsf{Set}$ to be a universe),
which induces for every functor in
$\mathsf{Fun}(\Delta^{\text{op}},\mathsf{Cat})$ 
a functor in $\mathsf{Fun}(\Delta^{\text{op}},\mathsf{Set})$.
We mean to say the latter map is constant.
\end{remark}

\begin{proof}
We need to construct a fully faithful functor
$$
F:\mathsf{Cat}_\Delta \to \mathsf{Fun}(\Delta^{\text{op}},\mathsf{Cat})
$$
whose image is the subcategory of functors 
whose simplicial sets of objects are constant.

To a $\mathsf{Set}_\Delta$-enriched category $C$ 
associate the functor $F(C)$ which maps $[n]$ 
to the category $C_n$, which is defined as follows.
The objects of $C_n$ are the objects of $C$ for all $n$.
(Notice that once we define the functor completely,
this property will make it indeed a functor whose simplicial
sets of objects are constant.) 
For $a,b$ in $C$, the morphisms of $C_n$ 
are $\Hom(a,b)_n$. To a map $f:[n] \to [m]$ in $\Delta^{\text{op}}$,
define $F(C)$ to give the functor of $C_m$ to $C_n$ that fixes
all objects and maps a map in $\Hom(a,b)_m$ 
to the induced map $\Hom(a,b)_n$ by the presheaf
$\Hom(a,b)$.

Now let's define how $F$ acts on morphisms. 
(This definition is just what it should be, but let's go over it.)
Choose two
$\mathsf{Set}_\Delta$-enriched categories $C,D$ 
and consider their corresponding functors $F(C),F(D)$.
Fix a morphism $\varphi \in \Hom_{\mathsf{Cat}_\Delta}(C,D)$.
Define a morphism (of presheaves of categories) 
$F(\varphi):F(C) \to F(D)$
defined as a collection of maps $F(C)_n \to F(D)_n$
given on objects by $\varphi$ and on morphisms
also given by $\varphi$, using that $\varphi$ is a morphism
of $\mathsf{Cat}_\Delta$ to ensure naturality.

Functoriality of $F$ follows from functoriality of each $\varphi$ as in the
previous paragraph.

Now let's confirm that $F$ is faithful, that is, it induces injections
on the Hom sets. Suppose $\varphi,\psi \in \Hom_{\mathsf{Cat}_\Delta}(C,D)$ 
are such that $F(\varphi)=F(\psi)$. By definition of $F(\varphi)$ and
 $F(\psi)$, it is immediate that $\varphi$ and $\psi$ coincide on objects.
In fact, it is also immediate that they coincide on morphism and as
simplicial sets by definition.

To prove $F$ is fully faithful we only need to check surjectivity of
the induced maps in Hom sets. Pick a morphism of presheaves
of categories, denote it
$F(\varphi)$, between two presheaves of categories $F(C)$ and  $F(D)$,
both of whose simplicial sets of objects are constant, namely
two sets $C$ and $D$.
Then we can define two $\mathsf{Set}_\Delta$-enriched categories,
which we also denote by $C$ and $D$,
by defining their objects to be the sets $C$ and $D$, and their morphisms to be
the collections of all the induced morphisms by $F(C)$ and $F(D)$ coming from
morphisms of $\Delta^{\text{op}}$.
Then it is immediate that the set $\Hom(C,D)$ is indeed a simplicial set.
Thus $C,D \in \mathsf{Cat}_{\mathsf{Set}_\Delta}$. Further, we
can define a morphism $\varphi \in \Hom_{\mathsf{Cat}_\Delta}(C,D)$ 
which maps on objects as any of the induced maps by the morphism of presheaves
of categories we started with (since both of the simplicial sets of objects
of the corresponding categories are constant!)
and on morphisms as well (any morphism of $C$ was defined as the
induced map by $F(C)$ coming from a map of $\Delta^{\text{op}}$).
It is clear that this morphism is mapped to $F(\varphi)$ under $F$.
\end{proof}

\begin{exercise}[1.3]
\label{exercise-N-is-fully-faithful}
Show that $N:\mathsf{Cat}\to\mathsf{Set}_\Delta$ is fully faithful.
\end{exercise}

\begin{proof}
We need to show that for any categories $A,B$,
$\Hom(A,B)=\text{Fun}(A,B)$ is in ``bijection'' with
$\Hom_{\mathsf{Set}_\Delta}(NA,NB)$.
Recall that
$NA$ is the presheaf that maps $[n]$ to
the set of composible sequence of $n$ morphisms in $A$.
Then to a functor $F:A \to B$ we associate the
map that sends a sequence of $n$ morphisms in $A$ 
to the respective sequence of $n$ morphisms in $B$ 
after applying $F$ to each object and map.

Conversely, given a morphism in $\mathsf{Set}_\Delta$
from $NA$ to $NB$
we can reconstruct a functor from $A$ to $B$
 by interpreting objects of $A$ as $NA_0$ and 
maps as $NA_1$.
\end{proof}

\section{Exercises of Rune, Chapter 2}
\label{section-exercises-of-Rune-Chapter-2}

\begin{definition}
\label{definition-path-space}
For points $x,y \in X$, the {\it path space} $X(x,y)$ 
is the pullback
$$
\xymatrix{
X(x,y)\ar[r]\ar[d]\ar@{}[dr]|-{\lrcorner}&\{x\}\ar[d]\\
\{y\}\ar[r]&X.
}
$$
\end{definition}

\begin{exercise}[2.1]
\label{exercise-2.1}
Assuming that pushouts exist and have the expected
universal property,
show that $\Omega^n_xX \simeq \mathsf{Map}_*(S^n,X)$,
where the $n$-th sphere is the pushout
$$
S^n:=* \amalg_{S^{n-1}}*,
$$
and the space of pointed maps is the pullback
$$
\xymatrix{
\mathsf{Map}_*(S^n,X)\ar[r]\ar[d]\ar@{}[dr]|-{\lrcorner}
&\mathsf{Map}(S^n,X)\ar[d]\\
\{x\}\ar[r]&\mathsf{Map}(*,X).
}
$$
\end{exercise}


\begin{proof}
Since both
$\Omega^1_xX$ and $\mathsf{Map}_*(S^1,X)$
are pullbacks (and pullbacks
are unique up to homotopy),
it's enough to show that they are
the pullback of the same diagram
(up to homotopy).

First notice that $\mathsf{Map}(*,X)\simeq X$ 
in an obvious way:
we identify a map $* \to X$ 
with the image of $*$.
To identify $\mathsf{Map}(S^n,X)$ 
with $\{x\} \simeq \mathsf{Map}(*,X)$
pick a map $* \to X$.
Now consider the universal 
property of pushouts:
$$
\xymatrix{
S^{n-1} \ar[r] \ar[d] & \ * \ar[d] \ar[rrdd] & & \\
\ * \ar[r] \ar[rrrd] & \ S^n=* \amalg_{S^{n-1}} \, * \ar@{-->}[rrd]^{\exists !}& & \\
& & & X
}
$$
\end{proof}

\begin{exercise}[2.2]
\label{exercise-2.2}
Use the 5-lemma to show that given a commutative triangle
$$
\xymatrix{
X \ar[rd]_{p} \ar[rr]_{f} && Y \ar[dl]^{q} \\
& B
}
$$
the morphism $f$ is an equivalence if and only if
the induced maps on the fibres
$p^{-1}(b)\to q^{-1}(b)$ are equivalences for all $b \in B$.
\end{exercise}

\begin{proof}
For the converse implication,
$$
\xymatrixcolsep{.7em}
\xymatrix{
\cdots \ar[r]&\pi_{n+1}(B,b)\ar[r]\ar[d]^{\simeq}
&\pi_n(p^{-1}(b),x)\ar[r]\ar[d]^{\simeq}
&\pi_n(X,x)\ar[r]\ar@{.>}[d]^{\simeq }
&\pi_n(B,b)\ar[r]\ar[d]^{\simeq}
&\pi_{n-1}(p^{-1}(b),x)\ar[d]^{\simeq}\ar[r]&\cdots\\
\cdots\ar[r]&\pi_{n+1}(B,b)\ar[r]&\pi_n(q^{-1}(b),f(x))\ar[r]&\pi_n(Y,f(x))\ar[r]&
\pi_n(B,b)\ar[r]&\pi_{n-1}(q^{-1}(b),f(x))\ar[r]&\cdots
}
$$
and for the forward implication just do the same with
the map $\pi_n(p^{-1}(b),x)\to \pi_n(q^{-1},f(x))$ in the center.
\end{proof}

\begin{exercise}[2.3]
\label{exercise-2.3}
{\bf Note:} I think this problem
is formulated wrong in the book
(there's no $g$ in the diagram!),
so I think this is the correct version:

A commutative square
$$
\xymatrix{
X'\ar[r]\ar[d]_{f'}&Y'\ar[d]^{f}\\
X\ar[r]_{g}&Y
}
$$
is a pullback if and only if for every $x \in X$,
the induced map on fibres is an equivalence.
\end{exercise}

\begin{proof}
The proof is using Corollary 2.1.23 that such a 
square is a pullback if and only if
for every $x \in X$ the induced map on fibres
is an equivalence. This reduces everything
to a situation completely analogous to 
Exercise \ref{exercise-2.2} and
we conclude by 5-lemma.
\end{proof}

\begin{exercise}[2.4]
\label{exercise-2.4}
Suppose we have a commutative diagram
$$
\xymatrix{
X\ar[r]\ar[d]&X'\ar[d]\ar[r]& X''\ar[d]\\
Y\ar[r]&Y'\ar[r]&Y''.
}
$$
\begin{enumerate}
\item If the right and composite squares are both pullbacks,
then so is the left-hand square.
\item If $\pi_0Y \to \pi_0Y'$ is surjective
and the left and composite squares
are both pullbacks,
then so is the right-hand square.
\end{enumerate}
\end{exercise}

\begin{proof}
\begin{enumerate}
\item To use the last exercise
we only need to show that $Y\to Y''$ is an equivalence.
But this is what Corollary 2.1.17: the 3-for-2 property.
\item I'm not sure why is that condition on
$\pi_0Y \to \pi_0Y'$… looks like the same argument should work.
\end{enumerate}
\end{proof}

\begin{exercise}[2.9]
\label{exercise-2.9}
Show that if $X$ is an $\infty$-groupoid,
then so is $\mathsf{Fun}(\mathcal{C},X)$ 
for any $\infty$-category $\mathcal{C}$.
\end{exercise}


\bibliography{my}
\bibliographystyle{amsalpha}

\end{document}

