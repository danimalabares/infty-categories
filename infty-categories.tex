\input{preamble}

\begin{document}

\title{Infty-categories}
\maketitle

\phantomsection
\label{section-phantom}
\hfill
\href{http://github.com/danimalabares/infty-categories}
{github.com/danimalabares/infty-categories}

\tableofcontents

\begin{definition}
\label{definition-discrete-fibration}
\begin{reference}
\url{https://ncatlab.org/nlab/show/discrete+fibration} 
\end{reference}
A functor $F:C \to B$ is a {\it discrete fibration} if for every object
$c$ in $C$ and every morphism of the form $g: b \to F(c)$ in $B$ 
there is a unique morphism $h:d \to c$ in $C$ 
such that $F(h)=g$.
\end{definition}

\begin{exercise}
\label{exercise-discrete-fibrations-over-category-are-presheaves}
Prove that discrete fibrations over a category $C$ correspond to presheaves over
$C$.
\end{exercise}

\begin{proof}
First suppose that we are given a presheaf $X$ on $C$. Define a discrete
fibration $F:C/X \to C$ by $(a,s)\mapsto a$ on objects and
mapping a morphism $f:(a,s)\to (b,t)$ in $C/X$ to the
corresponding morphism $a \to b$ in $C$.
To show $F$ is a discrete fibration 
let $g:b\to a$ be a morphism in $C$. 
Consider $g^*=X(g):X_a\to X_b$,
and the section $g^*s$ of $X_b$.
Then the
morphism $h:(b,g^*s)\to (a,s)$ is the only one mapping to $g$ under $h$.

For the converse let $F:B \to C$ be a discrete fibration over $C$.
To define a presheaf $X:C^{\text{op}}\to \text{Sets}$ let $c \in \Ob$.
We want to assign the set of objects in $B$ mapped to $c$ under $F$,
though it is not clear that this is a set…
\end{proof}

\begin{exercise}
\label{exercise-simplicial-sets-category-has-internal-Hom}
Show that the simplicial sets category  $\text{Set}_\Delta$ has internal 
Hom $S^T$ for simplicial sets $S$ and $T$, given by
$$
(S^T)_n:=\Hom_{\text{Set}_\Delta}(T\times\Delta^n,S)
$$
\end{exercise}

\begin{proof}
We need to show, as in \cite[Notation 1.1.13]{Cisinsky},
that $(S^T)_n$ is right adjoint to 
the functor $- \times S$.
That is,
$$
\Hom_{\text{Set}_\Delta}(U,S^T)\cong \Hom_\text{Set}_\Delta(U \times S,T)
$$
According to \cite[Theorem 1.1.3 and Remark 1.1.11]{Cisinsky},
we find that $(S^T)_n=\Hom_{\text{Set}_\Delta}(h_n \times S,T)$
where $h_n=\Hom(-,n)=\Delta_n$.
\end{proof}

\bibliography{my}
\bibliographystyle{amsalpha}

\end{document}

