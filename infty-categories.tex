\input{preamble}

\begin{document}

\title{Infty-categories}
\maketitle

\phantomsection
\label{section-phantom}
\hfill
\href{http://github.com/danimalabares/infty-categories}
{github.com/danimalabares/infty-categories}

\tableofcontents

\section{Other exercises}
\label{section-other-exercises}

\begin{definition}
\label{definition-discrete-fibration}
\begin{reference}
\url{https://ncatlab.org/nlab/show/discrete+fibration} 
\end{reference}
A functor $F:C \to B$ is a {\it discrete fibration} if for every object
$c$ in $C$ and every morphism of the form $g: b \to F(c)$ in $B$ 
there is a unique morphism $h:d \to c$ in $C$ 
such that $F(h)=g$.
\end{definition}

\begin{exercise}
\label{exercise-discrete-fibrations-over-category-are-presheaves}
Prove that discrete fibrations over a category $C$ correspond to presheaves over
$C$.
\end{exercise}

\begin{proof}
First suppose that we are given a presheaf $X$ on $C$. Define a discrete
fibration $F:C/X \to C$ by $(a,s)\mapsto a$ on objects and
mapping a morphism $f:(a,s)\to (b,t)$ in $C/X$ to the
corresponding morphism $a \to b$ in $C$.
To show $F$ is a discrete fibration 
let $g:b\to a$ be a morphism in $C$. 
Consider $g^*=X(g):X_a\to X_b$,
and the section $g^*s$ of $X_b$.
Then the
morphism $h:(b,g^*s)\to (a,s)$ is the only one mapping to $g$ under $h$.

For the converse let $F:B \to C$ be a discrete fibration over $C$.
To define a presheaf $X:C^{\text{op}}\to \text{Sets}$ let $c \in \Ob C$.
We assign the set (for now I won't justify why this is a set)
of objects in $B$ mapped to $c$ under $F$.
To define the correspondence on morphisms, 
consider a map $f:c \to d$ in $C^{\text{op}}$.
In other words, we have a map in  $C$ of the form $f^{\text{op}}:d \to c$.
Then to any object in $b$ such that $F(b)=c$,
by definition of discrete fibration,
we have a unique morphism of $B$ of the form
$h:r \to b$ such that $F(h)=f^{\text{op}}$.
In particular this means that $F(r)=d$.
This gives a function from $X(c)$ to $X(d)$.
This situation is described in the following diagram:

$$
\xymatrix{
X(d)\ni&r \ar@{.>}[d]_{\exists!h}\ar@{|->}[r]& d=F(r)
\ar[d]^{f^{\text{op}}=F(h)}\\
X(c)\ni& b\ar@{|->}[r]_F&c=F(b)
}
$$




To check functoriality of $X$ defined in the previous paragraph
suppose that $f:c\to d$ and $g:d\to e$ are two morphisms in $C^{\text{op}}$.
Like before, we have maps $f^{\text{op}}:d\to c$ and $g^{\text{op}}:e \to d$.
$$
\xymatrix{
X(e)\ni& q\ar@{.>}[d]_{\exists !j}\ar@{|->}[r]& e=F(q)
\ar[d]^{g^{\text{op}}=F(j)}\\
X(d)\ni&r \ar@{.>}[d]_{\exists!h}\ar@{|->}[r]& d=F(r)
\ar[d]^{f^{\text{op}}=F(h)}\\
X(c)\ni& b\ar@{|->}[r]_F&c=F(b)
}
$$
on the other hand, $gf:c \to e$ gives by the
same construction a unique map $k:\hat{q}  \to b$
such that $F(k)=f^{\text{op}}g^{\text{op}}$.
To check that $\hat{q}=q$, observe that
by functoriality of $F$, we have
$F(hj)=F(h)F(j)=f^{\text{op}}g^{\text{op}}=F(k)$.
By uniqueness of $k$, we conclude that
$k=hj$ and thus $q=\hat{q}$.
\end{proof}

\section{Exercises of Rune, Chapter 1}
\label{section-exercises-of-Rune-Chapter-1}

Here's my progress so far on the exercises in \cite{Rune}, Chapter 1.

\begin{exercise}[Observation 1.4.7]
\label{exercise-simplicial-sets-category-has-internal-Hom}
Show that the simplicial sets category  $\mathsf{Set}_\Delta$ has internal 
Hom $S^T$ for simplicial sets $S$ and $T$, given by
$$
(S^T)_n:=\Hom_{\mathsf{Set}_\Delta}(T\times\Delta^n,S)
$$
\end{exercise}

\begin{proof}
We need to show, that $S^T$ is the internal Hom in the category
$\mathsf{Set}_\Delta$. Different notations for the internal Hom
are $\text{Map}(-,-)$, $\underline{\Hom}(-,-)$.
It must be right adjoint to 
the functor $U \times -$.
That is,
$$
\Hom_{\text{Set}_\Delta}(U \times S,T)\cong\Hom_{\mathsf{Set}_\Delta}(U,S^T)
$$
(I think) I understand the statement correctly
but I don't understand how to apply
\cite[Theorem 1.1.10 (Kan)]{Cisinsky} nor \cite[Remark 1.1.11]{Cisinsky}
to prove it.


\end{proof}

\begin{exercise}[1.1]
\label{exercise-relation-is-equivalence-relation}
If $S$ is a Kan complex, then the relation defining
$\pi_0S$ is an equivalence relation.
\end{exercise}

\begin{proof}
\begin{enumerate}
\item (Reflexivity.) Let $a \in S_0$. Consider the composition
$$
\xymatrixrowsep{0.5}
\xymatrix{
[0]\ar[r]^{d_0}& [1]\ar[r]^{f_0}& [0]\\
0\ar@{|->}[r]& 1\ar@{|->}[r]&0
}
$$
since this gives the identity we must have
$$
\xymatrixrowsep{0.5}
\xymatrix{
S_0\ar[r]^{S(f_0)}& S_1\ar[r]^{S(d_0)}& [0]\\
a\ar@{|->}[r]& S(f_0)(a)\ar@{|->}[r]&a
}
$$
but we can replace $d_0$ by $d_1$ and we'd still get the identity,
so that $S(d_0)$ also maps $S(f_0)(a)$ to $a$. In other words, 
for any  $a \in S_0$ the 1-simplex $S(f_0)(a)$ is the desired one.
\item (Symmetry.) Let $a,b \in S_0$…
\end{enumerate}
\end{proof}

Rather informally, I understand a category $C$ to be
an {\it enriched category over $D$} if for any objects
$c,d$ in $C$, $\Hom(c,d)$ is an object of $D$.
Compositions of morphisms exist and are associative,
and there is an identity morphism for every object $c$ in $C$.
(See \url{https://ncatlab.org/nlab/show/enriched+category} 
for a formal definition.)

\begin{exercise}[1.2]
\label{exercise-enriched-simplicial-set-categories-are-subcategory-of-Fun}
Show that $\mathsf{Cat}_\Delta$ can be described as the full
subcategory of $\text{Fun}(\Delta^{\text{op}},\mathsf{Cat})$
containing the functors whose simplicial sets of objects
are constant.
\end{exercise}

\begin{proof}
To a $\mathsf{Set}_\Delta$-enriched category $C$ 
associate the functor which maps $[n]$ 
to the category $C_n$ defined as follows.
The objects of $C_n$ are the objects of $C$ for all $n$.
For $a,b$ in $C$, the morphisms of $C_n$ 
are $\Hom(a,b)_n$. To complete the definition of our
functor in $\text{Fun}(\Delta^{\text{op}},\mathsf{Cat})$ 
we must specify where to send a map $f:[n] \to [m]$ in $\Delta^{\text{op}}$.
It must go to a functor of $C$ to $C$ that fixes
all objects and maps a map in $\Hom(a,b)_n$ 
to the induced map $\Hom(a,b)_m$ by the presheaf
$\Hom(a,b)$.

Now we need to show that every functor whose simplicial sets
of objects is constant 
(i.e. constant functor $\Delta^{\text{op}}\to\mathsf{Cat}$)
can be expressed this way. Suppose $F$ is such a functor
with value $C$. Associate the $\mathsf{Set}_\Delta$-enriched
category $C$ with $\Hom(a,b)_n$ given by
$\Hom_{F([n])}(a,b)$.

(Hopefully my understanding of the term ``full subcategory'' is
not wrong?)
\end{proof}

\begin{exercise}[1.3]
\label{exercise-N-is-fully-faithful}
Show that $N:\mathsf{Cat}\to\mathsf{Set}_\Delta$ is fully faithful.
\end{exercise}

\begin{proof}
We need to show that for any categories $A,B$,
$\Hom(A,B)=\text{Fun}(A,B)$ is in ``bijection'' with
$\Hom_{\mathsf{Set}_\Delta}(NA,NB)$.
Recall that
$NA$ is the presheaf that maps $[n]$ to
the set of composible sequence of $n$ morphisms in $A$.
Then to a functor $F:A \to B$ we associate the
map that sends a sequence of $n$ morphisms in $A$ 
to the respective sequence of $n$ morphisms in $B$ 
after applying $F$ to each object and map.

Conversely, given a morphism in $\mathsf{Set}_\Delta$
from $NA$ to $NB$
we can reconstruct a functor from $A$ to $B$
 by interpreting objects of $A$ as $NA_0$ and 
maps as $NA_1$.
\end{proof}

\bibliography{my}
\bibliographystyle{amsalpha}

\end{document}

